\pagebreak
\subsection{Testplan}
Er zijn heel verschillende mogelijkheden voor het maken neurale netwerk. Het is moeilijk om te zien wat een positieve of negatieve invloed heeft. Daarom wordt elk getrainde neurale netwerk vergeleken, om te kijken welke eigenschappen samen het beste resultaat geeft. Om te komen tot het beste resultaat hebben we de volgende manieren om te testen: kijken naar de optimale errorwaarde, netwerk moet zo min mogelijk crashen, gemakkelijk herstellen na een crash en een parcours als snelste voltooien. Deze opties zullen verder uitgewerkt worden. \\\\
Voor elk parcours is een data beschikbaar van een goed, presterende controller voor Torcs. De errorwaarde voor het getrainde netwerk en een parcours kan binnen millisecondes berekent worden. Met de errorwaardes kan je gemakkelijk zien of het neurale netwerk goed getraind is en met deze data is het eerste verschil tussen het neurale netwerk te zien. \\\\
Het neurale netwerk kan ook getest worden in torcs, torcs heeft twee verschillende manieren om te runnen: text mode of gui mode. De text mode geeft beste laptijd, hoogste snelheid, schade en de totale race. In de textmode zijn alleen resultaten te zien, het voordeel is dat hiervoor weinig tijd nodig is van de tester en deze data veel meer zegt dat errorwaardes. Deze test wordt uitgevoerd waar het neurale netwerk wordt beoordeeld op de volgende criteria:
\begin{itemize}
\item De tijd waarop de auto 1 lap rijdt op de geselecteerde baan
\item Het aantal schaden dat de auto heeft aan het einde van 1 lap
\item De max snelheid die een auto behaald tijdens 1 lap
\end{itemize}
\noindent De gui mode wordt gebruikt om te zien hoe het neurale netwerk de auto crasht en de auto zichzelf terugzet op de weg. Door de damage uit de vorige test kan er al worden geconstateert of een auto crasht. Elke crashsituatie is uniek, dus het is niet zeker of het neurale netwerk zichzelf terug kan plaatsen op de weg. Het beste neurale netwerk zal nooit crashen, maar om zeker te weten of het zichzelf uit een crash kan halen, moet er een crashsituatie gecre\"erd worden. Er kan voor gekozen worden om de simulatie gelijk te laten crashen en kijken hoe het neurale netwerk dit oppakt. Of te races op een dirttrack, hier zal de auto snel tegen een muur rijden, echter is het niet zeker of de unstuck methode hetzelfde werkt op de dirt track dan op een road track.\\\\
Gedurende het testen wordt er gekeken welke testmethodes het beste werken en daar zullen we dan ook uiteindelijk mee gaan testen. In het kopje resultaten zal dit verder worden uitgelegd. 
