\documentclass{article}
\usepackage{graphicx} 
\usepackage[dutch]{babel}
\newcommand\tab[1][1cm]{\hspace*{#1}}
\usepackage{cite}
\usepackage[skip=2pt]{caption}
\graphicspath{ {resources/} } 
\usepackage{hyperref}
\begin{document}
\sffamily
\begin{titlepage}
  \centering
    \vfill
    {\bfseries\Huge
      Tinlab Machine Learning \\
      Groepsverslag
        \vskip2cm
      }
      {\bfseries\Large
      	Thomas Alakopsa\\
      	{ \bfseries\normalsize
      	0911723\\
      	}
      }
      {\bfseries\Large
      	Alex de Ridder\\
      	{ \bfseries\normalsize
      	0937558\\
      	}
      }
      {
        \bfseries\normalsize
        \vskip2cm
        \today
    }    
    \vfill
    \includegraphics[width=4cm]{logohr.png}
    \vfill
    \vfill
\end{titlepage}
\newpage

\section{Samenvatting}

\input{verklarendewoordenlijst.tex}

\section{Inleiding}
Voor Tinlab Machine Learning wordt de verworven kennis toegepast door een intelligente controller te maken voor race simulatie Torcs. In plaats van zelf aan de knoppen te zitten en de auto te besturen, zal er een programma geschreven worden die aan de hand van getrainde modellen en binnenkomende data zelfstandig de auto bestuurd. 


\section{Projectopzet}

\section{Vooronderzoek}

\subsection{Neuraal netwerk}
Neurale netwerken zijn een reeks algoritmen die losjes gemodelleerd zijn van het menselijke brein. Een artificial brein dat gemaakt is uit een hele grote reeks artificial neurons.
%Wat is artificial? Goed om uit te leggen en in woordenlijst te douwen :)

\subsubsection{Perceptrons}
Een van de meest fundamenteele artifical neuron types is een perceptron. Perceptronen zijn een belangrijk onderdeel van een neuraal netwerk en kennis hierover is nodig om een neuraal netwerk te begrijpen. Een perceptron pakt verschillende binary inputs:$x_{1}, x_{2},....x_{n}$ en produceerd een enkele binaire output. Je kan het zien als een functie die beslissingen voor je neemt, door verschillende factoren tegen elkaar te wegen en uiteindelijk met ja of nee te antwoorden.
\begin{figure}[h!]
\centering
\includegraphics[scale=0.5]{perceptron2.png}
\caption{uitleg}
\label{peceptron2}
\end{figure}
\linebreak
In afbeelding \ref{peceptron2} is een perceptron te zien die 3 variabelen als input neemt: $x_{1}, x_{2}$ en $x_{3}.$ Bij all deze waardes word een gewicht(weight) toegekend($w_{n}$). Deze waarde geeft aan hoe belangrijk de input is voor deze neuron. De output van de neuron is de som van alle resultaten bij elkaar. $\sum_{j}w_{j}x_{j}$ en deze waarde vergelijken met een gekozen randwaarde(threshold) om de output the berekenen. In een meer wiskundige term:
%\begin{equation*} %tTODO write function
%\begin{rcases}
%        0  if \sum_{j}w_{j}x_{j}+b $\leq$ threshold $\\$
%        1  if \sum_{j}w_{j}x_{j}+b > threshold
%\end{rcases} 
%$\text{output}$
%\end{equation*}
 \begin{equation}
    y(x_1,\ldots,x_n) = f(w_1x_1 + w_2x_2 + \ldots + w_nx_n)  \label{per-eq}
  \end{equation}

\noindent Je kan de output van een neuron beinvloeden door te spelen met de weights en thresholds. Door een input zijn weight te vergroten of de threshold te verlagen kan er hele andere resultaten uit het model komen.\\
\newline
Het is duidelijk dat de perceptron niet een compleet model is over hoe mensen hun beslissingen nemen. Maar het voorbeeld illustreert hoe een perceptron verschillende soorten bewijs kan afwegen om beslissingen te nemen. Daarom is het aannemelijk dat een complex netwerk van perceptrons vrij subtiele beslissingen zou moeten kunnen nemen.
\begin{figure}[h!]
\centering
\includegraphics[scale=0.5]{perceptron3.png}
\caption{uitleg}
\label{perceptron3}
\end{figure}
\newline
In afbeelding \ref{perceptron3} is een netwerk te zien, waar de eerste laag van perceptrons in het netwerk, drie simpele beslissingen neemt door de functie in vergelijking \ref{per-eq} uit te voeren. Naast de eerste laag zit er nu ook een tweede die de outputs van de eerste laag als input neemt. Op deze manier kan een perceptron in de tweede laag een beslissing nemen op een complexer en abstracter niveau dan perceptrons in de eerste laag. Deze complexiteit en abstractheid word verhoogd per extra laag dat je toevoegd. Op deze manier kan een meerlaags netwerk van perceptrons, zeer geavanceerde beslissingen nemen.\\
\newline
De volgende stap is om ons netwerk zelf lerend te maken. Om dit te doen moet je kleine aanpassingen kunnen maken aand de weights en de biases. Deze kleine aanpassingen moet daarna ook een klein effect hebben op de output van het neurale netwerk. Echter dat is niet wat er gebeurt met perceptronen want deze heeft maar 2 outputs, een 1 en een 0. Een kleine aanpassing zal daarom niks doen of de hele uitkomst van de perceptron omdraaien. Je kan niet probleem omzeilen door een ander types neurons te gebruiken ,zoals de Sigmoid en tanh neurons.

\subsubsection{tanh neurons}
sigmoid en tanh neurons lijken erg op perceptrons alleen de manier hoe de ouput berekend word is anders. 


\subsection{python}

\subsection{Encog}

\section{Methode}
Aan het begin van het project zijn meerdere datasets gemaakt door de input en output te loggen van een al goed rijdend systeem. Er is voor gekozen om deze datasets te gebruiken. \\\\ 
Om te communiceren met de "Torcs server" kan er gebruikt gemaakt worden van een client in Java of C++. Er is voor gekozen om de Java client te gebruiken, met als voornaamste reden dat er op github~\cite{java-client} een opzet te vinden is om een Neural Netwerk te schrijven. Deze versie maakt gebruik van een algoritme, waar geen Machine Learning voor wordt gebruikt. \\\\
Het eindproduct zal een neuraal netwerk zijn dat getraind is door middel van een dataset en verifi\"erd door andere datasets. Aanpassingen in de instellingen van het neurale netwerk zullen uiteindelijk leiden tot het beste progamma. Het neurale netwerk zal tijdens het trainen bij bepaalde iteraties opgeslagen worden en bij vroegtijdig stopzetten kan de training ook weer hervat worden vanaf de laatst opgeslagen iteratie. \\\\
Het getrainde neurale netwerk kan worden geverifi\"erd worden op twee manieren, namelijk met de trainingsets of door het \textit{live} te runnen op een baan. Bij het het live runnen is duidelijk te zien hoe het neurale netwerk reageert op bochten. In de resultaten zal dan het verband uitgelegd worden tussen de instellingen van het neurale netwerk en de real-time uitvoering. 

\section{Resultaten}

\section{Conclusie}

\newpage

\bibliographystyle{plain}
%\bibliography{IEEEabrv,references}
\bibliography{references}
\end{document}